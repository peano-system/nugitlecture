\documentclass[../main]{subfiles}
% \usepackage{amsmath}

\begin{document}
\section*{平成31年度名古屋大学文系大問2(1)の解答}
\[
\begin{cases}
x_{n+1}&=x_n-k(y_n+y_{n+1}) \\
y_{n+1}&=y_n+k(x_n+x_{n+1})
\end{cases}
\]
について,1行目を2行目に代入・整理すると
\[
y_{n+1}=\frac{2k}{1+k^2}x_n+\frac{1-k^2}{1+k^2}y_n
\]
が得られる.
さらに,この式を$x_{n+1}=x_n-k(y_n+y_{n+1})$に代入・整理すると
\[
x_{n+1}=\frac{1-k^2}{1+k^2}x_n-\frac{2k}{1+k^2}y_n
\]
が得られる.ここで
\begin{align*}
\frac{1-k^2}{1+k^2}
&=(1-\tan^2(\frac{\alpha}{2}))\cos^2(\frac{\alpha}{2})
=\cos^2(\frac{\alpha}{2})-\sin^2(\frac{\alpha}{2})
=\cos\alpha,\\
\frac{2k}{1+k^2}
&=2\tan(\frac{\alpha}{2})\cos^2(\frac{\alpha}{2})
=2\sin(\frac{\alpha}{2})\cos(\frac{\alpha}{2})
=\sin\alpha
\end{align*}
であるから
\[
x_{n+1}=x_n\cos\alpha-y_n\sin\alpha,
y_{n+1}=x_n\sin\alpha+y_n\cos\alpha
\]
となる.
$\mathrm{P}_0$の座標は$(1,0)$であるから
\[
x_1=x_0\cos\alpha-y_0\sin\alpha=\cos\alpha,
y_1=x_0\sin\alpha+y_0\cos\alpha=\sin\alpha
\]
である.また
\[
x_2=x_1\cos\alpha-y_1\sin\alpha=\cos^2\alpha-\sin^2\alpha=\cos2\alpha,
y_2=x_1\sin\alpha+y_1\cos\alpha=2\sin\alpha\cos\alpha=\sin2\alpha
\]
である.
従って,
$\mathrm{P}_1$の座標は$(\cos\alpha,\sin\alpha)$,
$\mathrm{P}_2$の座標は$(\cos2\alpha,\sin2\alpha)$
である.
\end{document}