\documentclass[main]{subfiles}

\begin{document}
%ここへいつも通り本文を打ち込めばよいようです.
%マクロが使いたい場合は main のプリアンブルへ書き込んでください.
%問題を書くか確認する,というか問題を書かなくていいか確認する.
\setcounter{prob}{3}
\begin{prob}

\end{prob}
\begin{pf}
(1) $2x^2+x+3 =2(x^2+1)+x+1$ より,$[2x^2+x+3] = x+1$ である.

次に,$x^5-1 = (x^3-x)(x^2+1)+x-1$ より,$[x^5-1] = x-1$ である.

最後に,$[2x^2+x+3][x^5-1] = (x+1)(x-1) = x^2-1 = (x^2+1)-2$ より,
$[[2x^2+x+3][x^5-1]] = -2$ である. 

(2) [・] の定義より, 
\begin{eqnarray*}
A(x) = \exists P(x)(x^2+1)+[A(x)], \\
B(x) = \exists Q(x)(x^2+1)+[B(x)]
\end{eqnarray*}
と書ける.ただし,$P(x)$, $Q(x)$ は整式である.

このとき次が成り立つ.
\begin{eqnarray*}
A(x)B(x) &=& \{P(x)(x^2+1)+[A(x)]\}\{Q(x)(x^2+1)+[B(x)]\} \\
				&=& \{P(x)Q(x)(x^2+1)+(P(x)[B(x)]+Q(x)[A(x)])\}(x^2+1) \\ 
				                                                        &+& \quad [A(x)][B(x)]
\end{eqnarray*}
$\{\}$ の中身は整式だから,
\begin{equation*}
[A(x)][B(x)] = [[A(x)][B(x)]]
\end{equation*}
が成り立つ.

(3)


(4)
\end{pf}
%%%%%%%%%%%%%%%%%%%%%%%%%%%%%%%%%%%%%%%%%%%%%%%%%%%%%%%%%%%%%%%%%%%%
\begin{prob}

\end{prob}
\begin{pf}
(1) 
\begin{eqnarray*}
I &:= \int^1_0\sin^2(\pi x) dx, \\
I_1 &:= \int^1_{-1}\frac{\sin^2(\pi x)}{1+e^x}dx, \\
I_2 &:= \int^1_{-1}\frac{e^x\sin^2(\pi x)}{1+e^x}dx
\end{eqnarray*}
とおく.

まず,$I = \frac{1}{2}$ を示す.
\begin{eqnarray*}
\sin^2(\pi x) &=& \frac{1-\cos(2\pi x)}{2}
\end{eqnarray*}
より,
\begin{eqnarray*}
I &=& \frac{1}{2}\int^1_0 1-\cos(2\pi x)dx \\
	&=& \frac{1}{2}[x-\frac{1}{2\pi}\sin(2\pi x)]^1_0 \\
	&=& \frac{1}{2}
\end{eqnarray*}
と積分できる.

次に,$I_1 = \frac{1}{2}$ を示そう.
\begin{eqnarray*}
I_1 + I_2 &=& \int^1_{-1}\sin^2(\pi x) \frac{1+e^x}{1+e^x}dx \\
			&=& 2\int^1_0\sin^2(\pi x)dx \\
			&=& 2I = 1
\end{eqnarray*}
次に, 
\begin{eqnarray*}
I_2 -I_1 &=& \int^1_{-1}\sin^2(\pi x) \frac{-1+e^x}{1+e^x}dx \\
&=& \int^1_{-1}\sin^2(\pi x) 
					\frac{e^\frac{x}{2}-e^{-\frac{x}{2}}}{e^\frac{x}{2}+e^{-\frac{x}{2}}}dx \\
\end{eqnarray*}
特に被積分関数は奇関数であるから,$I_2 = I_1$. これより,$I_1 =\frac{1}{2}$
\end{pf}
%%%%%%%%%%%%%%%%%%%%%%%%%%%%%%%%%%%%%%%%%%%%%%%%%%%%%%%%%%%%%%%%%%%%%
\begin{prob}

\end{prob}
\begin{pf}

\end{pf}
\end{document}