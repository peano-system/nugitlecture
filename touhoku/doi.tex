\documentclass[main]{subfiles}

\begin{document}
%ここへいつも通り本文を打ち込めばよいようです.
%マクロが使いたい場合は main のプリアンブルへ書き込んでください.
%問題を書くか確認する,というか問題を書かなくていいか確認する.
\setcounter{prob}{3}
\begin{prob}

\end{prob}
\begin{pf}
$(1)$ $2x^2+x+3 =2(x^2+1)+x+1$ より,$[2x^2+x+3] = x+1$ である.

次に,$x^5-1 = (x^3-x)(x^2+1)+x-1$ より,$[x^5-1] = x-1$ である.

最後に,$[2x^2+x+3][x^5-1] = (x+1)(x-1) = x^2-1 = (x^2+1)-2$ より,
$[[2x^2+x+3][x^5-1]] = -2$ である. 

$(2)$ $[・]$ の定義より, 
\begin{eqnarray*}
A(x) = \exists P(x)(x^2+1)+[A(x)], \\
B(x) = \exists Q(x)(x^2+1)+[B(x)]
\end{eqnarray*}
と書ける.ただし,$P(x)$, $Q(x)$ は整式である.

このとき次が成り立つ.
\begin{eqnarray*}
A(x)B(x) &=& \{P(x)(x^2+1)+[A(x)]\}\{Q(x)(x^2+1)+[B(x)]\} \\
				&=& \{P(x)Q(x)(x^2+1)+(P(x)[B(x)]+Q(x)[A(x)])\}(x^2+1) \\ 
				                                                        &+& \quad [A(x)][B(x)]
\end{eqnarray*}
$\{\}$ の中身は整式だから,
\begin{equation*}
[A(x)][B(x)] = [[A(x)][B(x)]]
\end{equation*}
が成り立つ.

$(3)$
\begin{eqnarray*}
(x\sin\theta+\cos\theta)^2 &=& \sin^2\theta(x^2+1)+
										\cos^2\theta-\sin^2\theta+2x\sin\theta\cos\theta \\
\end{eqnarray*}
よって,
\begin{equation*}
[(x\sin\theta+\cos\theta)^2] = x\sin2\theta+\cos2\theta 
\end{equation*}
を得る.

$(4)$ 
\begin{eqnarray*}
r := \sqrt{a^2+b^2} \\
\cos\theta := \frac{b}{r} \\
\sin\theta := \frac{a}{r}
\end{eqnarray*}
と定義する.このとき,
\begin{equation*}
ax+b= r(x\sin\theta+\cos\theta)
\end{equation*}
と書ける.すると,$(3)$ より
\begin{eqnarray*}
[(ax+b)^4]&=&r^4[[(x\sin\theta+\cos\theta)^2][(x\sin\theta+\cos\theta)^2]] \\
&=&r^4[(x\sin2\theta+\cos2\theta)^2] \\
&=&r^4(x\sin4\theta+\cos4\theta)
\end{eqnarray*}
これが $-1$ に等しいので,
\begin{itemize}
\item $r^4 = 1$ \\
\item $\sin4\theta = 0$ \\
\item $\cos4\theta = -1$
\end{itemize}
となる.特に $r \in \R$ より,これは
\begin{equation*}
r = 1,\quad \theta = -\frac{\pi}{4}+\frac{n\pi}{2}, \quad (n \in \Z)
\end{equation*}
という条件と同値である.
これを $(a,b)$ に直すと,
\begin{equation*}
(a,b)=(\pm \frac{1}{\sqrt{2}}, \pm \frac{1}{\sqrt{2}}) ,\quad (\text{複号任意})
\end{equation*}
を得る.
\end{pf}
%%%%%%%%%%%%%%%%%%%%%%%%%%%%%%%%%%%%%%%%%%%%%%%%%%%%%%%%%%%%%%%%%%%%
\begin{prob}

\end{prob}
\begin{pf}
$(1) $
\begin{eqnarray*}
I &:= \int^1_0\sin^2(\pi x) dx, \\
I_1 &:= \int^1_{-1}\frac{\sin^2(\pi x)}{1+e^x}dx, \\
I_2 &:= \int^1_{-1}\frac{e^x\sin^2(\pi x)}{1+e^x}dx
\end{eqnarray*}
とおく.

まず,$I = \frac{1}{2}$ を示す.
\begin{eqnarray*}
\sin^2(\pi x) &=& \frac{1-\cos(2\pi x)}{2}
\end{eqnarray*}
より,
\begin{eqnarray*}
I &=& \frac{1}{2}\int^1_0 1-\cos(2\pi x)dx \\
	&=& \frac{1}{2}[x-\frac{1}{2\pi}\sin(2\pi x)]^1_0 \\
	&=& \frac{1}{2}
\end{eqnarray*}
と積分できる.

次に,$I_1 = \frac{1}{2}$ を示そう.
\begin{eqnarray*}
I_1 + I_2 &=& \int^1_{-1}\sin^2(\pi x) \frac{1+e^x}{1+e^x}dx \\
			&=& 2\int^1_0\sin^2(\pi x)dx \\
			&=& 2I = 1
\end{eqnarray*}
次に, 
\begin{eqnarray*}
I_2 -I_1 &=& \int^1_{-1}\sin^2(\pi x) \frac{-1+e^x}{1+e^x}dx \\
&=& \int^1_{-1}\sin^2(\pi x) 
					\frac{e^\frac{x}{2}-e^{-\frac{x}{2}}}{e^\frac{x}{2}+e^{-\frac{x}{2}}}dx \\
\end{eqnarray*}
特に被積分関数は奇関数であるから,$I_2 = I_1$. これより,$I_1 =\frac{1}{2}$

$(2)$
\begin{equation*}
a := \int^1_{-1}f(t)dt, \quad b := \int^1_{-1}e^tf(t)dt
\end{equation*}
とおく.すると,
\begin{eqnarray*}
(1+e^x)f(x) &=& \sin^2(\pi x)+\int^1_{-1}(e^x-e^t+1)f(t)dt \\
&=&\sin^2(\pi x)+(e^x+1)a - b
\end{eqnarray*}
と書ける.
すると,辺々積分することで
\begin{eqnarray*}
a + b = \int^1_{-1}(1+e^x)f(x)dx = \int^1_{-1}\sin^2(\pi x)dx 
											+ a \int^1_{-1}(1+e^x)dx -2b \\
= 1 + a(2 + e -e^{-1}) -2b
\end{eqnarray*}
同様に,
\begin{eqnarray*}
a = \int^1_{-1}f(x)dx = \int^1_{-1}\frac{\sin^2(\pi x)}{1+e^x}dx
									+ 2a -b\int^1_{-1}\frac{1}{1+e^x}dx \\
									= \frac{1}{2} + 2a - b
\end{eqnarray*}
を得る.これらの式を用いて,
\begin{itemize}
\item $a = \frac{1}{2(e-e^{-1}-2)}$ \\
\item $b = \frac{1}{2(e-e^{-1}-2)} + \frac{1}{2}$
\end{itemize}
が成り立つ.
これを元の式に代入して, 
\begin{equation*}
f(x) = \frac{\sin^2(\pi x)}{1+e^x} + \frac{1}{2(e-e^{-1}-2)} 
	- \frac{1}{1+e^x} \left\{ \frac{1}{2(e-e^{-1}-2)} + \frac{1}{2} \right\}
\end{equation*}
を得る.
\end{pf}
%%%%%%%%%%%%%%%%%%%%%%%%%%%%%%%%%%%%%%%%%%%%%%%%%%%%%%%%%%%%%%%%%%%%%
\begin{prob}

\end{prob}
\begin{pf}
$(1)$ $n+1$ 回目の試行が終わったとき, 赤玉が $2$ 個である場合は,
\begin{enumerate}
\item $n$ 回目に赤玉が $2$ 個で, 白玉を引く.
\item $n$ 回目に白玉が $1$ 個で, 赤玉を引く.
\end{enumerate}
の二つに分けられる. 
白玉と赤玉の数の和は常に $10$ であることに注意すると, 
\begin{equation*}
p(n+1, 2) = \frac{7}{10}p(n,2) + \frac{4}{10}p(n,1)
\end{equation*}
が成り立つ.

$(2)$, $(3)$ 
$(1)$ と同様に $p(n+1,1)$, $p(n+1,0)$ を $p(n,2)$, $p(n,1)$, $p(n,0)$ で表すと, 
次の表示を得る.
\begin{equation*}
\begin{pmatrix}
p(n+1,2) \\
p(n+1,1) \\
p(n+1,0)
\end{pmatrix} 
= \frac{1}{10}
\begin{pmatrix}
7 & 4 & 0 \\
0 & 6 & 5 \\
0 & 0 & 5
\end{pmatrix}
\begin{pmatrix}
p(n,2) \\
p(n,1) \\
p(n,0)
\end{pmatrix}
\end{equation*}
ここで, 
\begin{equation*}
A =  \frac{1}{10}
\begin{pmatrix}
7 & 4 & 0 \\
0 & 6 & 5 \\
0 & 0 & 5
\end{pmatrix}
\end{equation*}
と置くと, 
\begin{equation*}
\begin{pmatrix}
p(n,2) \\
p(n,1) \\
p(n,0)
\end{pmatrix} 
= A^n
\begin{pmatrix}
p(0,2) \\
p(0,1) \\
p(0,0)
\end{pmatrix} 
= A^n
\begin{pmatrix}
0 \\
0\\
1
\end{pmatrix} 
\end{equation*}
と書くことができる.

あとは $A^n$ を求めればよい.
いま, 
\begin{equation*}
P = 
\begin{pmatrix}
1 & -4 & 10\\
0 & 1 & -5\\
0 & 0 & 1
\end{pmatrix}
\end{equation*}
と置くと, $P^{-1}AP=\rm{diag}(7,6,5)$ が分かるので, 
\begin{equation*}
A^n = P \rm{daig}(7^n,6^n,5^n) P^{-1}
\end{equation*}
を計算して,
\begin{eqnarray*}
p(n,1) = \left(\frac{1}{10}\right)^n \left\{ 6^n5 -5^n5 \right\} \\
p(n,2) = \left(\frac{1}{10}\right)^n \left\{ 3^n10 -6^n20 + 5^n10 \right\}
\end{eqnarray*}
を得る.
\end{pf}
\end{document}