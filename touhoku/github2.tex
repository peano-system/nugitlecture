\documentclass[main]{subfiles}
%\usepackage{amsmath}
%\usepackage{amssymb}
%\usepackage{amsthm}
%\newtheorem{problem}{問題}
%\newtheorem{answer}{解答}
%\renewcommand{\labelenumi}{(\arabic{enumi})}
%\numberwithin{equation}{answer}

 

\begin{document}
%%%%%%%%%%%%%%%%%%%%%%%%%%%%%%%%%%%%%%%%%%%%%%%%%%%%%%%%%%%%%%%%%%%%%%%%%%%%%
\begin{answer}
$x=s, \, t$での接線を考えると
    \begin{align}
    &y=(\cos s)x+\sin s - s \cos s\notag\\
    &y=(\cos t)x+\sin t - t \cos t \label{1.1}%%%%%%%%%label1.1%%%%%%%%%%%%%
\end{align}
$2$つの接線が直交するので$\cos s \cdot \cos t = -1.$ \\
$-1 \le \cos s, \cos t \le 1$であり,$\, \cos s \ge \cos t$としても一般性を失わないので
\begin{align*}
\, \cos s= 1 &, \, \cos t =-1\\
\therefore s=2n\pi &, \,  t=(2k+1)\pi. \quad (n, k \in \mathbb{Z} )
\end{align*}
これと$(\ref{1.1})$より,
\begin{align*}
y = x-2n\pi = -x+(2k+1)\pi \\
\therefore x=\left( n+k+\frac{1}{2} \right)\pi , \quad x=\left( -n+k+\frac{1}{2} \right)\pi
\end{align*}

\end{answer}
 
%%%%%%%%%%%%%%%%%%%%%%%%%%%%%%%%%%%%%%%%%%%%%%%%%%%%%%%%%%%%%%%%%%%%%%%%%%%%%%%%%%
\begin{answer}
まず$\log$の定義より
\begin{align}
x-n>0 \text{かつ}2n-x>0 \Longleftrightarrow n<x<2n \label{2.1} %%%%%%%label2.1%%%%%%
\end{align}
この範囲で
\begin{align}
&\log_{a}(x-n) > \frac{1}{2} \log_{a} (2n-x) \notag \\
\Longleftrightarrow
&\log_{a}(x-n)^{2} > \log_{a} (2n-x) \notag \\
\Longleftrightarrow
&\begin{cases}
a>1 \\
(x-n)^{2} > 2n-x
\end{cases}
\text{or} \,\,
\begin{cases}
0 <a <1 \\
(x-n)^{2} < 2n-x
\end{cases} \notag \\
\Longleftrightarrow
&\begin{cases}
a>1 \\
x^{2} - (2n-1) x + n^{2} -2n >0
\end{cases}
\text{or} \,\,
\begin{cases}
0 <a <1 \\
x^{2} - (2n-1) x + n^{2} -2n <0
\end{cases} \label{2.2} %%%%%%%%%%%%label{2.2}
\end{align}

 

%%%%%%%%%%%%%%%%%%%%%%%%%%%%%%%%%%%%%%%%%%%%%%%%%%%%%%%%%%%%%%%%%%%%%%%%%%%%%%%%%%%%%%%%
$(1) $
$a>1$のとき$(\ref{2.1})$と$(\ref{2.2})$に$n=6$を代入し
\begin{align*}
6< x <12 \, \text{かつ} \, x^{2} -11x +24>0 \Longleftrightarrow 8 < x <12
\end{align*}
$0<a<1$のとき同様に
\begin{align*}
6< x <12 \, \text{かつ} \, x^{2} -11x +24<0 \Longleftrightarrow 6 < x < 8
\end{align*}
以上より求める整数$x$は
\begin{align*}
a>1 \text{のとき} x=9, 10, 11 \quad 0<a<1 \text{のとき} x=7
\end{align*}

 

%%%%%%%%%%%%%%%%%%%%%%%%%%%%%%%%%%%%%%%%%%%%%%%%%%%%%%%%%%%%%%%%%%%%%%%%%%%%%%%%%%%%%%
$(2) $
$f(x)=x^{2} - (2n-1) x + n^{2} -2n$とおく.すると$f(x)$は
\begin{align*}
f(n)= -n <0 , \quad f(2n)=n^{2} >0
\end{align*}
なる下に凸な二次関数である.\\
$n<x<2n$で$(\ref{2.2})$を満たす$x$が存在する必要十分条件を考える.\\
$(a) \, a>1$のとき \\
求める条件は$f(2n-1)>0$となるときなので,
\begin{align*}
f(2n-1)=n(n-2) >0 \Longleftrightarrow n>2
\end{align*}

$(b) \, 0<a<1$のとき \\
求める条件は$f(n+1)<0$となるときなので,
\begin{align*}
f(n+1)=-n+2 < 0 \Longleftrightarrow n>2
\end{align*}
$(a), \, (b)$よりいずれの場合でも求める必要十分条件は$n>2$.
\end{answer}

%%%%%%%%%%%%%%%%%%%%%%%%%%%%%%%%%%%%%%%%%%%%%%%%%%%%%%%%%%%%%%%%%%%%%%%%%%%%%%%%%%%%%%

\begin{answer}
$(1)$
$x_{n+1} - x_{n} = x_{n}^{2} \ge 0$より,数列$\{ x_{n} \}$は単調増加する.よって
\begin{align*}
&x_{n+1} - x_{n} = x_{n}^{2} \ge x_{1}^{2} = a^{2} \,\, (\because a>0)\\
&\therefore \, x_{n+1} \ge x_{n} +a^{2}
\end{align*}
これを繰り返し用いると
\begin{align*}
x_{n} \ge x_{1} +a^{2} (n-1) = a +a^{2} (n-1) \to \infty \quad (n \to \infty)
\end{align*}
以上より数列$\{ x_{n} \}$は発散する.\\

%%%%%%%%%%%%%%%%%%%%%%%%%%%%%%%%%%%%%%%%%%%%%%%%%%%%%%%%%%%%%%%%%%%%%%%%%%%%%%%%%%%%%
$(2)$数学的帰納法により示す.\\
$(a)n=1$のとき\\
$-1<a<0$つまり$-1<x_{1}<0$より満たす.\\
$(b) -1<x_{k} <0$と仮定する.\\
$\displaystyle x_{k+1} = x_{k} + x_{k}^{2} =\left(x_{k}+\frac{1}{2} \right)^{2} -\frac{1}{4}$より,
仮定の範囲では$-\displaystyle\frac{1}{4} \le x_{k+1} < 1.$\\
よって$-1<x_{k+1}<1$を満たす.\\
$(a) \,(b)$より示された. \\
\\

%%%%%%%%%%%%%%%%%%%%%%%%%%%%%%%%%%%%%%%%%%%%%%%%%%%%%%%%%%%%%%%%%%%%%%%%%%%%%%%%%%%%
$(3)$
$-1<a<0$のとき$(2)$より$x_{n+1}, \, x_{n}, \, x_{n}+1 \neq 0$なので,漸化式の逆数をとると
\begin{align*}
\frac{1}{x_{n+1}} = \frac{1}{x_{n}+x_{n}^{2}}
= \frac{1}{x_{n}} - \frac{1}{x_{n}+1}
< \frac{1}{x_{n}} -1 \,\,\, \left(\because (2)-1<x_{n}<1\right)
\end{align*}
これを繰り返し用いると
\begin{align*}
\frac{1}{x_{n}} < \frac{1}{x_{1}} + (-1) \cdot (n-1) = \frac{1}{a} -(n-1) \to -\infty \quad (n \to \infty)
\end{align*}
以上より$\displaystyle\lim_{n \to \infty}\frac{1}{x_{n}} = - \infty$なので,
$\displaystyle\lim_{n \to \infty} x_{n} =0$.
\end{answer}

 

 

 

 

 

 

\end{document}