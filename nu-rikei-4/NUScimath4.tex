\documentclass[11pt]{jarticle}
\usepackage[a4paper,hmargin=20mm,vmargin=15pt]{geometry}
\usepackage{amsmath}
\usepackage{amsthm}
\usepackage{amssymb}
\newtheorem{lemma}{Lemma}
\newtheorem{theorem}{Theorem}
\pagestyle{plain}
\begin{document}
\title{}
\date{}
\maketitle
\paragraph{問題4.}
(1) $(a_1, a_2, a_3)$の組み合わせとして起こりうるものと
それらが持つサイクルは, 
\begin{align*}
  (1, 2, 3) \quad a_1=& 1,\quad a_2= 2,\quad a_3= 3 \\
  (1, 3, 2) \quad a_1=& 1,\quad a_2= 3 \to a_3= 2 \\
  (2, 1, 3) \quad a_1=& 2 \to a_2= 1,\quad a_3= 3 \\
  (2, 3, 1) \quad a_1=& 2 \to a_2= 3 \to a_3= 1 \\
  (3, 1, 2) \quad a_1=& 3 \to a_3= 2 \to a_2= 1 \\
  (3, 2, 1) \quad a_1=& 3 \to a_3= 1,\quad a_2= 2
\end{align*}
である. このうち, $(1,2,3), (1,3,2), (2,1,3), (3,2,1)$が長さ$1$のサイクルを持つ. 
従って求める確率は, 
\begin{equation*}
  \frac{4}{6} = \underline{\frac{2}{3}}
\end{equation*}
\end{document}