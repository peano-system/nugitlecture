\documentclass{jarticle}
\usepackage{amsmath}
\usepackage{amssymb}
\usepackage{amsthm}

\begin{document}

\subsection*{(2)の解答}
題意を満たすとき,2次方程式
\[
a x^2+bx+c = 0 
\]
は,自然数
\[a_1,a_2,c_1,c_2
\]を用いて
\[
(a_1 x-c_1)(a_2 x-c_2)=0
\]
と変形できる.このとき,
\[
\begin{cases}
a_1 a_2&=a \\
c_1 c_2&=c \\
a_1 c_2+a_2 c_1&=b
\end{cases}
\]
また,xは整数解であるので自然数$l$,$m$を用いて
\[
\begin{cases}
c_1&=l a_1\\
c_2&=m a_2
\end{cases}
\]
と表せられる.
これを初めの式に代入すると,
\[
\begin{cases}
a_1 a_2=a\\
(la_1)(ma_2)=lma_1a_2=lma=c\\
a_1(ma_2)+a_2(la_1)=(l+m)a_1a_2=(l+m)a=b
\end{cases}
\]
ここで,$a$,$b$,$c$が1から6の間をとる整数であることと、\\
$l$と$m$の値が入れ替わったときに$a$,$b$,$c$の値が変わらないことに気を付けて($l$,$m$)の組を求めると\\

$a$=1のとき\\
($l$,$m$)=(1,1),(1,2),(1,3),(1,4),(1,5),(2,2),(2,3)で\\
対応する($a$,$b$,$c$)はそれぞれ(1,2,1),(1,3,2),(1,4,3),(1,5,4),(1,6,5),(1,4,4),(1,5,6)

$a$=2のとき\\
($l$,$m$)=(1,1),(1,2)で\\
対応する($a$,$b$,$c$)はそれぞれ(2,4,2),(2,6,4)

$a$=3のとき\\
($l$,$m$)=(1,1)で\\
対応する($a$,$b$,$c$)は(3,6,3)\\

よって求める確率は
\[
\frac{10}{6^3}=\frac{5}{108}
\]


\end{document}