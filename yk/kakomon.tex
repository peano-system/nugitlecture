\documentclass[main]{subfiles}
%\usepackage{amsmath}
%\usepackage[dvipdfmx]{graphicx}
%\usepackage{tikz}
%\usepackage[dvipdfmx]{graphicx}
%\usepackage{tikz}
%\usepackage{amsmath, amssymb}
%\usepackage{type1cm}
\begin{document}
(2)
\begin{align*}
5.1^2=26.01\cdots
\\6.1^2=37.2\cdots
\\7.1^2=50.41\cdots
\\8.1^2=65.6\cdots
\\9.1^2=82.8\cdots
\\10.1^2=102.01\cdots
\\11.1^2=123.2\cdots
\\12.1^2=146.4\cdots
\end{align*}
以上より条件を満たす$n$は小さい順に
$$
26,37,50,65,82,101,102,122,123,145,\cdots
$$
よって145 \\

(別解法) \\
$\sqrt{n}$ の整数部分を $m$ とおく. \\
問題の条件から $\sqrt{n}$ は整数ではないので
\begin{align*}
 m < \sqrt{n} < m + 1 \\
 m^2 < n < (m+1)^2
\end{align*}
となるので, $n=m^2 + k$ のようになる. (ただし$k =1, 2, \cdots 2m$)
また小数点以下に関する条件から
\begin{align*}
0.01 \le \sqrt{n} -m < 0.1 \\
m + 0.01 \le \sqrt{n} < m + 0.1 \\
(m + 0.01)^2 \le n (=m^2 + k) < m + 0.1 \\
0.02m + 0.0001 \le k < 0.2m + 0.01
\end{align*}
ここで $0.02m + 0.0001 \ge 1$ とすると, $m \le 49.995$.
(1)によって一番小さい$n$ は $26 = 5^2 +1$ であった. (つまり$m=5$.) \\
一方で, $f(m) = 0.2m + 0.01 $ としておくと,
$f(5)= 1.01 , f(6) = 1.21 f(7) = 1.41,  f(8) = 1.61,  f(9) = 1.81,  f(10) = 2.01, f(11) = 2.21, f(12) = 2.41$
のようになる. つまり$m=5, 6, \cdots, 12$ のときの$k$ の個数は$1, 1, 1, 1, 1, 2, 2, 2$ のようになる.
求めるものは, 小さいものから10番目であるので小さいものから数えてゆくと,
$m=12$ で $k=1$ のときつまり$n=12^2 +1 = 145$. \\
以上から145 が答えである.
\end{document}