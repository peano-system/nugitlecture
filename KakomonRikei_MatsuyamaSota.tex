\documentclass[a4j, 11pt]{jarticle}
\usepackage{amsmath}
\usepackage{amssymb}
\usepackage{amsthm}

\begin{document}
\section*{2}
% (2)
\subsection*{(2)}
$\frac{\pi}{2} < \angle B'AC' < \pi$のとき, $\cos \angle B'AC < 0$なので,
$\overrightarrow{AB'} \cdot \overrightarrow{AC'} < 0$
であることを示せば良い.
\begin{align*}
	&\quad \overrightarrow{AB'} \cdot \overrightarrow{AC'} \\
	%
	&= (\overrightarrow{AB} + \overrightarrow{BB'}) \cdot
	(\overrightarrow{AC} + \overrightarrow{CC'}) \\
	%
	&= \overrightarrow{AB} \cdot \overrightarrow{AC}
	+ \overrightarrow{AB} \cdot \overrightarrow{CC'}
	+ \overrightarrow{BB'} \cdot \overrightarrow{AC}
	+ \overrightarrow{BB'} \cdot \overrightarrow{CC'} \\
	%
	&= 0 + (\overrightarrow{AB'} + \overrightarrow{B'B}) \cdot
	\overrightarrow{CC'} + \overrightarrow{BB'} \cdot
	(\overrightarrow{AC'} + \overrightarrow{C'C})
	+ \overrightarrow{BB'} \cdot \overrightarrow{CC'} \\
	%
	&= \overrightarrow{AB'} \cdot \overrightarrow{CC'}
	+ \overrightarrow{B'B} \cdot \overrightarrow{CC'}
	+ \overrightarrow{BB'} \cdot \overrightarrow{AC'}
	+ \overrightarrow{BB'} \cdot \overrightarrow{C'C}
	+ \overrightarrow{BB'} \cdot \overrightarrow{CC'} \\
	%
	&= - \overrightarrow{BB'} \cdot \overrightarrow{CC'}
	- \overrightarrow{BB'} \cdot \overrightarrow{CC'}
	+ \overrightarrow{BB'} \cdot \overrightarrow{CC'} \\
	%
	&= - \overrightarrow{BB'} \cdot \overrightarrow{CC'} \\
	%
	&= - |BB'| |CC'| \cos 0 < 0
	\quad (\because BB' \parallel CC', 
	\quad |BB'| > 0 \text{かつ} |CC'| > 0)
	%
\end{align*}
したがって, 
$\overrightarrow{AB'} \cdot \overrightarrow{AC'} < 0$が成り立つので, 
$\angle B'AC' > \frac{\pi}{2}$が成立する.
% (3)
\subsection*{(3)}
\end{document}