\documentclass[main]{subfiles}

\begin{document}

 $g(x) = F(x) - f(x)$ とする. 曲線 $y = g(x)$ が $x$ 軸と異なる3つの交点をもつためには,  
 $g(x)$ が2つの極値をもち, かつ極大値と極小値の積が負になればよい.  \par

微分積分学の基本定理より, 
 \begin{equation*}
  \frac{d}{dx}\int_{0}^{x} f(t) dt =  f(x)
 \end{equation*}
である. これより, 
 \begin{align*}
  g^{'}(x) &= f(x) - f^{'}(x)   \\
           &= x^2 + (a - 2)x - 2a \\
           &= (x + a)(x - 2).
 \end{align*}
よって, 求める条件は
 \begin{equation}
  a \neq -2
 \end{equation}
 \begin{equation}
  g(-a)g(2) < 0
 \end{equation}
で与えられる. 
\par
ここで, 
\begin{equation*}
0>g(-a)g(2)=(F(-a)-f(-a))(F(2)-f(2)).
\end{equation*}
さらに
\[ F(x) = \frac{1}{3}x^3 + \frac{1}{2} ax^2 - ax \]
なので
\begin{align}
g(-a) &= \frac{1}{6} a^3 +a^2 +a, \notag \\
g(2) &= -(a+\frac{4}{3}) \notag
\end{align}
がわかる. つまり
\[ g(-a) g(2) = -a(\frac{1}{6}a^2 + \frac{1}{2}a + 1)(a+\frac{4}{3}) \]
が負となるような$a$の範囲を求めて$a \neq -2$を考慮すれば, それがいま考えている問題に対する答えである. $g(-a)g(2)<0$となる$a$の範囲は
\[ a<-3-\sqrt{3} \ {\rm または} \ -\frac{4}{3} < a<  -3+\sqrt{3} \ {\rm または} \ a>0 \]
である. これは$a=-2$を含まないので$(1)$ $(2)$を同時に満たす$a$の範囲である. よって, これが答えとなる.




\end{document}
