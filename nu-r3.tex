\documentclass[11pt]{jarticle}
\usepackage[a4paper,hmargin=20mm,vmargin=15pt]{geometry}
\usepackage{amsmath}
\usepackage{amsthm}
\usepackage{amssymb}
%%%%%%%%%%%
\begin{document}
正の整数$n$の正の平方根$\sqrt{n}$の整数部分を$a$とおくと, $a \ge 1$である. このとき$\sqrt{n}$が整数でなく, 十進数における小数点第1位が0となり, かつ第2位が0でないための必要十分条件は
\begin{equation*}
 a+0.01 \le \sqrt{n} < a+0.1
\end{equation*}
となる. 各辺を2乗して$a^2$を引くと, $a \ge 1$から$0.02a+0.0001>0$となるため, 
\begin{equation}
 0 < 0.02a+0.0001 \le n-a^{2} < 0.2a+0.01
\label{eq:1}
\end{equation}
これを満たす$a$が存在する$n$の中で最小のものが求める$n$である. 
$n-a^{2}$は正整数であるため, $a$は$0.2a+0.01>1$を満たす必要がある. 
$0.2a+0.01$が単調増加であり, $0.2 \times 4+0.01=0.81<1$, $0.2 \times 5+0.01=1.01>1$から$a \ge 5$となることが必要である. 
また$n-a^2 \ge 1$より$n \ge a^2 +1$となり, $a \ge 5$から$n=26$が条件を満たす最小のものとなる. 
実際, 不等式(\ref{eq:1})に$a=5$, $n=26$を代入すると
\begin{equation*}
 0 < 0.1001 \le 1 < 1.01
\end{equation*}
となり条件を満たしていることが確認できる. 
\begin{flushright}
 答え$n=26$
\end{flushright}
\end{document}