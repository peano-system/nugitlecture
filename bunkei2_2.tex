\documentclass[a4j,12pt,draft]{jarticle}
\usepackage{amsmath}
\usepackage{amssymb}
\usepackage{amsthm}

\begin{document}
	\subsection*{(2)}
		$(x_n,y_n) = (cos n \alpha,sin n \alpha) \ \ -(*)$であることを数学的帰納法により示す。
		\paragraph{}
			$(x_0,y_0) = (1,0) = (cos0 \alpha,sin0 \alpha)$より$n =0$の時$(*)$は成立。
		\paragraph{}
			n = kの時$(*)$が成立すると仮定。すなわち$(x_k,y_k) = (cos k\alpha,sin k \alpha)$ \\
			(1)で導出した
				\begin{itemize}
					\item $x_{n+1} = (cos\alpha)x_n - (sin\alpha)y_n$
					\item $y_{n+1} = (sin\alpha)x_n + (cos \alpha)y_n$
				\end{itemize}
			に対しn = kを代入して
				\begin{eqnarray*}
					x_{k+1} &=&  (cos\alpha)x_k - (sin\alpha)y_k \\
							&=&cos\alpha cosk\alpha - sin\alpha sink\alpha \\ &=& 
							cos(k + 1)\alpha \\
				\end{eqnarray*}
				\begin{eqnarray*}
					y_{k+1} &=& (sin\alpha)x_k + (cos \alpha)y_k \\
						&=& sin\alpha cosk\alpha + cos\alpha sink \alpha \\
						&=& sin(k + 1)\alpha
				\end{eqnarray*}
			すなわちn = k+1の時も$(*)$が成立する。
			
		\paragraph{}
		以上より$(x_n,y_n) = (cosn\alpha,sinn\alpha)$
			


\end{document}