\documentclass[../main]{subfiles}
%
\begin{document}
 \setcounter{prob}{3}
 \prob
 $(4)$
 \begin{pf}
  \begin{enumerate}
   \item 長さ $n$ のサイクルを持つとき
   $n$ におけるサイクルの総数は $n!$ 個であり,
   長さ$n$ のサイクルを持つ場合の数は
   各 $a_i \quad (i = 1, 2, \cdots n)$ に対し,
   $a_i = i$ でなく, $a_i$ がすでに選んだ数字でなければよいので
   $ (n - 1) \cdot (n - 2) \cdots 1 = (n - 1)!$ 個となる.
   よって長さ $n$ のサイクルを持つ確率は
   \begin{equation*}
    \frac{(n - 1)!} {n!} = \frac{1} {n}
   \end{equation*}
  となる.
   \item 長さ $k \quad (\frac{n+1} {2} \le k \le n - 1)$ のサイクルを持つとき
  \end{enumerate}
  長さ $n - k$ 以下と長さ $k$ のサイクルを持つことになるので
  長さ $k$ のサイクルを持つ場合の数は
  どの  $n - k$ 個の $a_{i}$ が
  長さ $n - k$ 以下のサイクルを持つかで $_n C _{n - k}$ 通りあり,
  $n - k$ におけるサイクルの総数は $(n - k)!$ 個である.
  そして残りの$k$ 個の $a_i$ が長さ $k$ のサイクルとなればよく,
  その個数は $n$ の場合と同様に考え,  $(k - 1)!$ 個であるから
  $_n C _{n - k} \cdot (n - k)! \cdot (k - 1)!
  = n \cdot (n - 1) \cdots (k + 1) \cdot (k - 1)!$
  となる.
  よって長さ $k$ のサイクルを持つ確率は
   \begin{equation*}
    \frac{n \cdot (n - 1) \cdots (k + 1) \cdot (k - 1)!} {n!} = \frac{1} {k}
   \end{equation*}
  となる.
  \setcounter{prob}{3}
  よって, 求める確率 $p$ は \prob $(3)$ より,
  \begin{equation*}
    p = \sum _{k = \frac{n + 1} {2}} ^ {n} {\frac{1} {k}}
   > \log(n + 1) - \log \left(\frac{n + 1} {2} \right)
   =  \log(n + 1) - \log(n + 1) + \log2
   =\log2
  \end{equation*}
  よって示された.
 \end{pf}
\end{document}